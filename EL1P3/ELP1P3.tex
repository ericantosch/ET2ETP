\documentclass{article}

\usepackage{circuitikz} %Für die Schaltpläne
\usepackage[T1]{fontenc} 
\usepackage[utf8]{inputenc}
\usepackage{amsmath}
\usepackage{amssymb}
\usepackage{fancyhdr}
\usepackage{graphicx}
\usepackage{hyperref}
\usepackage{subcaption}
\usepackage{tikz}
\usepackage{../assets/scripts/tex/color-env}
\usepackage[ngerman]{babel}



    \usetikzlibrary{arrows}
    \usetikzlibrary{arrows.meta,topaths}
    \usetikzlibrary{bending}
    \usetikzlibrary{calc}
\title{Elektrotechnik 1 Praktikum 1}


\usepackage[
  includehead,
  headheight = 17mm,
  footskip = \dimexpr\headsep+\ht\strutbox\relax,
  tmargin = 0mm,
  bmargin = \dimexpr17mm+2\ht\strutbox\relax,
]{geometry}

\usepackage{anyfontsize}

\usepackage{xcolor}

\definecolor{DarkGreenBlue}{HTML}{264653}
\definecolor{LightGreenBlue}{HTML}{2A9D8F}
\definecolor{LightOrange}{HTML}{E9C46A}
\definecolor{DarkOrange}{HTML}{F4A261}
\definecolor{RedOrange}{HTML}{E76F51}
\definecolor{BrightRed}{HTML}{D62828}
\definecolor{DeepBlue}{HTML}{003049}



\pagestyle{fancy}
\fancyhead[L]{\leftmark}
\fancyhead[R]{}
\fancyfoot[L]{}
\fancyfoot[C]{\thepage}
\fancyfoot[R]{\includegraphics[scale=0.2]{../assets/images/haw.jpg}}
\renewcommand\headrulewidth{0.5pt}


\begin{document}



\begin{tikzpicture}[overlay,remember picture]
  \thispagestyle{empty}
  \fill[black!2] (current page.south west) rectangle (current page.north east);

  \begin{scope}[transform canvas ={rotate around ={45:($(current page.north west)+(-.5,-6)$)}}]

    \shade[rounded corners=18pt, left color=DarkGreenBlue, right color=LightGreenBlue] ($(current page.north west)+(-.5,-6)$) rectangle ++(9,1.5);

  \end{scope}

  \begin{scope}[transform canvas ={rotate around ={45:($(current page.north west)+(.5,-10)$)}}]

    \shade[rounded corners=18pt, left color=LightOrange,right color=DarkOrange] ($(current page.north west)+(0.5,-10)$) rectangle ++(15,1.5);

  \end{scope}

  \begin{scope}[transform canvas ={rotate around ={45:($(current page.north west)+(0.5,-10)$)}}]

    \shade[rounded corners=8pt, right color=DarkOrange, left color=LightOrange] ($(current page.north west)+(1.5,-9.55)$) rectangle ++(7,.6);

  \end{scope}

  \begin{scope}[transform canvas ={rotate around ={45:($(current page.north)+(-1.5,-3)$)}}]

    \shade[rounded corners=12pt, left color=DeepBlue!80, right color=DeepBlue!60] ($(current page.north)+(-1.5,-3)$) rectangle ++(9,0.8);

  \end{scope}

  \begin{scope}[transform canvas ={rotate around ={45:($(current page.north)+(-3,-8)$)}}]

    \shade[rounded corners=28pt, left color=BrightRed, right color=BrightRed!80] ($(current page.north)+(-3,-8)$) rectangle ++(15,1.8);

  \end{scope}

  \begin{scope}[transform canvas ={rotate around ={45:($(current page.north west)+(4,-15.5)$)}}]

    \shade[rounded corners=25pt, left color=RedOrange, right color=DarkOrange] ($(current page.north west)+(4,-15.5)$) rectangle ++(30,1.8);

  \end{scope}

  \begin{scope}[transform canvas ={rotate around ={45:($(current page.north west)+(13,-10)$)}},]

    \shade[rounded corners=22pt, left color=DeepBlue,right color=DarkGreenBlue] ($(current page.north west)+(13,-10)$) rectangle ++(15,1.5);

  \end{scope}

  \begin{scope}[transform canvas ={rotate around ={45:($(current page.north west)+(18,-8)$)}},]

    \shade[rounded corners=8pt, left color=DarkOrange] ($(current page.north west)+(18,-8)$) rectangle ++(15,0.6);

  \end{scope}

  \begin{scope}[transform canvas ={rotate around ={45:($(current page.north west)+(19,-5.65)$)}},]

    \shade[rounded corners=12pt, left color=RedOrange] ($(current page.north west)+(19,-5.65)$) rectangle ++(15,0.8);

  \end{scope}

  \begin{scope}[transform canvas ={rotate around ={45:($(current page.north west)+(20,-9)$)}}]

    \shade[rounded corners=20pt, left color=BrightRed, right color=BrightRed!80] ($(current page.north west)+(20,-9)$) rectangle ++(14,1.2);

  \end{scope}

  \draw[ultra thick,gray] ($(current page.center)+(5,2)$) -- ++(0,-3cm) node[midway,left=0.25cm,text width=5cm,align=right,black!75]{{\fontsize{25}{30} \selectfont \bf Elektronik 1\\[10pt] Praktikum 3}} node[midway,right=0.25cm,text width=6cm,align=left,orange]{{\fontsize{70}{86} \selectfont 2020}};

  \node at ($(current page.center)+(0,-4)$) {{\fontsize{60}{72} \selectfont Bipolar Transistor}};

  \node[text width=8cm,align=center] at ($(current page.center)+(0,-6.5)$) {{\fontsize{16}{20} \selectfont \textcolor{orange}{ \bf \today}} \\[3pt] Florian Tietjen\\[3pt] Eric Antosch};

\end{tikzpicture}
\newpage
\thispagestyle{empty}

\tableofcontents


\newpage


\section{Kennlinie eines npn-Transistors}
\begin{task}
  TBei dieser Aufgabe sollen die Kennlinienfelder eines BC546A npn-Transistors mithilfe von LTSpice simuliert und analysiert werden. Dazu sollen bestimmte Kenndaten aus den Kennlinienfeldern rausgelesen werden.
\end{task}

\begin{figure}[h]
  \centering
  \includegraphics[scale=0.76]{../assets/images/EL1P3/Schaltplan1.png}
  \caption{Schaltplan zur Messung der Kennlinienfelder}
  \label{fig:schalt1}
\end{figure}

\subsection{Ausgangskennlinie}
\label{sec:ausgangskennlinie}

Wir wollen zunächst das Ausgangskennlinienfeld des Transistors bestimmen. Dazu messen wir $I_{C} = 0...2mA$ und $U_{CE} = 0...10V$ für mindestens 5 verschiedene Basisströme und tragen diese
in LTSpice ab. Für die Basisströme benutzen wir eine variable Spannungsquelle mit einem hochohmigen Vorwiderstand aus der E24-Reihe. Wir wissen aus dem Datenblatt des BC546A, dass die Stromverstärkung bei ca. $B=200$ liegt, wir rechnen also:
\begin{equation}
  \label{eq:1}
  I_{B} = \frac{I_{CE}}{B} = \frac{2mA}{200} = 10\mu A.
\end{equation}
Wir wählen nun eine Spannungsquelle mit einem Maximum von 10V. Um das Maximum von $I_{B} = 10\mu A$ zu bekommen, wollen wir nun den Vorwiderstand unter Berücksichtigung von $U_{BE} = 0,7V$ bestimmen:
\begin{equation}
  \label{eq:2}
  R_{V} = \frac{U_{0}-U_{BE}}{I_{B}} = \frac{10V-0,7V}{10\mu A} = 930k\Omega.
\end{equation}

Wir setzen den Wert von $R_{V} = 930k\Omega$ aus den Widerständen $R_{V1} = 110k\Omega$ und $R_{V2} = 820k\Omega$ zusammen (siehe (\ref{fig:schalt1})). Wir variieren nun mit DC-Sweep den Wert von $UCE$ zwischen $0V-10V$ in einem li
\newpage

\section{Verstärker in Emitterschaltung}

\begin{task}
  TBei dieser Aufgabe soll ein Operationsverstärker mit dem npn-Transistor BC546A näher untersucht werden.
  Insbesondere der Einfluss der Eingangsspannung auf den Operationsverstärker wird mit LTSpice analysiert.
\end{task}

\begin{figure}[h]
  \centering
  \includegraphics{../assets/images/EL1P3/aufbau aufgabe2.jpg}
  \caption{Versuchsaufbau}
  \label{fig:schalt2}
\end{figure}

\subsection{Arbeitspunkteinstellung}
Für die Arbeitspunkteinstellung wird nur die Gleichspannungsquelle $U_B$ betrachtet und die jeweiligen Maschen ohne Kondensatoren, da diese sich beim Gleichstrom wie unenedliche große Widerstände verhalten.
Der Kollektorstrom $I_{C0}$ beträgt 2mA.

\begin{align*}
  &I_B = 10\mu A\\  
  &I_{R1}=10\cdot I_B = 100\mu A\\
  &I_{R2}= 9\cdot I_B = 90\mu A\\
  &U_{R2} = U_{CE} + U_{RE} = 0,7V + (2mA\cdot 1k\Omega) = 2,7V\\
  &R_{2} = \frac{2,7V}{90\mu A} = 30k\Omega\\
  &U_{R1} = U_B - U_{R2} = 15V - 2,7V = 12,3V\\
  &R_{1} = \frac{12,3V}{100\mu A}=123k\Omega
\end{align*}

Da Widerstände aus der E24-Reihe verwendet werden sollen, wird für $R_1$ 120k eingesetzt. \\
Die berechneten Werte werden mithilfe von Simulationen in LTSpice überprüft. Es finden sich die berechneten Werte mit minimalen Abweichungen wieder.

\newpage
\subsection{Wechselspannungsverstärkung}

\begin{figure}[h]
  \centering
  \includegraphics{../assets/images/EL1P3/aufbau 2 2.JPG}
  \caption{Aufbau in LTSpice}
\end{figure}

\begin{figure}[h]
  \centering
  \includegraphics[scale=0.5]{../assets/images/EL1P3/leerlauf aufgabe 2.JPG}
  \caption{Zeitlicher Verlauf der Leerlaufspannung und Eingangsspannung}
\end{figure}


Die Leerlaufspannungsverstärkung beträgt $\frac{U_{out}}{U_{in}}=\frac{2,9V}{14,14mV} =190 $

\subsection{Bestimmung des Ein- und Ausgangswiderstandes (Messung)}

Zur Bestimmung des differentiellen Ausgangswiderstand wird ein Lastwidestand so dimensioniert, dass über ihn genau die Hälfte der Leerlaufspannung abfällt.
\\ Dies ist ungefähr der Fall bei $R_L = 3k\Omega$\\
Nun wird das selbe Verfahren für den differentiellen Eingangswiderstand angewendet, dabei wird ein Widerstand direkt an die Quelle U1 in Reihe mit dem Kondensator C1 geschaltet. 
\\ Hierbei ergibt sich die Hälfte der Leerlaufspannung bei $R = 2,5k\Omega$


\end{document}
