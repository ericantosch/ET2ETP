\documentclass{article}

\usepackage{circuitikz}
\usepackage[T1]{fontenc} 
\usepackage[UTF8]{inputenc}
\usepackage{amsmath}
\usepackage{amssymb}
\usepackage{fancyhdr}
\usepackage{graphicx}
\usepackage{hyperref}
\usepackage{tikz}
  \usetikzlibrary{arrows}
  \usetikzlibrary{shapes}
  \usetikzlibrary{arrows.meta,topaths}
  \usetikzlibrary{bending}
  \usetikzlibrary{calc}
\usepackage{anyfontsize}
\usepackage{sectsty}
\usepackage{../assets/scripts/tex/color-env}
\usepackage{anyfontsize}
\usepackage{xcolor}
\definecolor{DarkGreenBlue}{HTML}{264653}
\definecolor{LightGreenBlue}{HTML}{2A9D8F}
\definecolor{LightOrange}{HTML}{E9C46A}
\definecolor{DarkOrange}{HTML}{F4A261}
\definecolor{RedOrange}{HTML}{E76F51}
\definecolor{BrightRed}{HTML}{D62828}
\definecolor{DeepBlue}{HTML}{003049}



\usepackage[ngerman]{babel}
\title{Elektrotechnik 1 - Praktikum 3}


\usepackage[
  includehead,
  headheight = 17mm,
  footskip = \dimexpr\headsep+\ht\strutbox\relax,
  tmargin = 0mm,
  bmargin = \dimexpr17mm+2\ht\strutbox\relax,
]{geometry}





\pagestyle{fancy}
\fancyhead[L]{\leftmark}
\fancyhead[R]{}
\fancyfoot[L]{}
\fancyfoot[C]{\thepage}
\fancyfoot[R]{\includegraphics[scale=0.2]{../assets/images/haw.jpg}}
\renewcommand\headrulewidth{0.5pt}


\begin{document}


\thispagestyle{empty}
\begin{tikzpicture}[remember picture,overlay]

  \fill[DeepBlue] (current page.south west) rectangle (current page.north east);

  \begin{scope}

    \foreach \i in {2.5,...,22}
      {
        \node[rounded corners, DeepBlue!90,draw ,regular polygon, regular polygon sides=6, minimum size=\i cm, ultra thick] at ($(current page.west)+(2.5,-5)$) {} ;
      }

  \end{scope}

  \node[rounded corners,fill=DeepBlue!95,text =DeepBlue!5,regular polygon,regular polygon sides=6, minimum size=2.5 cm,inner sep=0,ultra thick] at ($(current page.west)+(2.5,-5)$) {\LARGE \bfseries 2020};

  \foreach \i in {0.5,...,22}
    {
      \node[rounded corners,DeepBlue!90,draw,regular polygon,regular polygon sides=6, minimum size=\i cm,ultra thick] at ($(current page.north west)+(2.5,0)$) {} ;
    }

  \foreach \i in {0.5,...,22}
    {
      \node[rounded corners,DeepBlue!98,draw,regular polygon,regular polygon sides=6, minimum size=\i cm,ultra thick] at ($(current page.north east)+(0,-9.5)$) {} ;
    }

  \foreach \i in {12}
    {
      \node[fill = DeepBlue,rounded corners,draw=DeepBlue,regular polygon,regular polygon sides=6, minimum size=\i cm,ultra thick] at ($(current page.south east)+(-0.2,-0.45)$) {} ;
    }


  \foreach \i in {21,...,6}
    {
      \node[DeepBlue!95,rounded corners,draw,regular polygon,regular polygon sides=6, minimum size=\i cm,ultra thick] at ($(current page.south east)+(-0.2,-0.45)$) {} ;
    }

  \node[left,DeepBlue!5,minimum width=0.625*\paperwidth,minimum height=3cm, rounded corners] at ($(current page.north east)+(0,-9.5)$){{\fontsize{25}{30} \selectfont \bfseries ET2 - Praktikum 5}};

  \node[left,DeepBlue!10,minimum width=0.625*\paperwidth,minimum height=2cm, rounded corners] at ($(current page.north east)+(0,-11)$){{\huge \textit{Resonanz}}};

  \node[left,DeepBlue!5,minimum width=0.625*\paperwidth,minimum height=2cm, rounded corners] at ($(current page.north east)+(0,-13)$){{\Large \textsc{Florian Tietjen\hspace{0.5cm}Eric Antosch}}};

\end{tikzpicture}

\newpage
\thispagestyle{empty}

\tableofcontents


\newpage


\section{Vorbereitung}
Als Vorbereitung der folgenden Versuche werden vorher einige Kenngrößen eines Serienresonanzkreies berechnet.
Die Serienschaltung besteht dabei aus einer Induktivität L = 100 mH und einem Gleichstrom-Drahtwiderstand $R_L = 10 \Omega$ . Der Sinus-Generator
hat einen Innenwiderstand von $R_i = 50\Omega$. Die angegebene Größe der Induktivität gilt nur für Frequenzen f < 1kHz.

\subsection{Berechnung der erforderlichen Kapazität}
Die Formel zur Berechnung der Resonanzfrequenz wird nach der Kapazität C umgestellt.
\begin{align*}
  \omega_r &= \frac{1}{\sqrt{LC}}\\
  C &= \frac{1}{{w_r}^2L}
\end{align*}
Mit der Formel werden nun die Kapazitäten verschiedener Frequenzen ausgerechnet:
\begin{center}

  \begin{tabular}{|c|c|c|c|c|c|}    
    \hline
    Freqeunz               & $100 Hz$                   & $500Hz$                   & $1kHz$            & $5kHz$          & $10kHz$              \\
    \hline
    Kapazität              & $25,33\mu F$                      & $1,013\mu F$                     & $253,3nF$            & $1,013nF$           & $2,53nF$              \\
    \hline
  \end{tabular}

\end{center}

\subsection{Berechnung der Gütefaktoren und Bandbreiten}
Nun werden die zugehörigen Gütefaktoren Q und Bandbreiten $\Delta$f für jeweils den Gesamtverlustwiderstand R$_i$ + R$_L$ und den Spulen-Drahtwiderstand R$_L$ allein berechnet.
Für den Gesamtverlustwiderstand gilt:
\begin{align*}
  R_{ges} &= R_i + R_L= 50\Omega + 10\Omega = 60\Omega
\end{align*}
Die Formel der Güte einer Serienschaltung lautet:
\begin{align*}
  Q &= \frac{1}{R}\cdot\sqrt{\frac{L}{C}}
\end{align*}
Für die Bandbreite $\Delta$f ermittelt sich aus:
\begin{align*}
  B &= \frac{f_r}{Q}
\end{align*}
Für verschiedene Frequenzen ergeben sich dann folgende Güten und Bandbreiten:

\begin{center}

  \begin{tabular}{|c|c|c|c|c|c|}    
    \hline
    Freqeunz               & $100 Hz$                   & $500Hz$                   & $1kHz$            & $5kHz$          & $10kHz$              \\
    \hline
    $Q_{R_{ges}}$              & $25,33\mu F$               & $1,013\mu F$                     & $253,3nF$            & $1,013nF$           & $2,53nF$              \\
    \hline
    $Q_L$              & $100 Hz$                   & $500Hz$                   & $1kHz$            & $5kHz$          & $10kHz$              \\
    \hline
    $\Delta f_{R_{ges}}$              & $25,33\mu F$                      & $1,013\mu F$                     & $253,3nF$            & $1,013nF$           & $2,53nF$              \\
    \hline
    $\Delta f_L$              & $25,33\mu F$                      & $1,013\mu F$                     & $253,3nF$            & $1,013nF$           & $2,53nF$              \\
    \hline
  \end{tabular}

\end{center}



\newpage
\section{Resonanz - Eingeschwungener Zustand}
\begin{task}
  IIm ersten Versuch soll der Amplitudengang von $\hat{u}_C$ in der abgebildeten Schaltung mit dem Oszilloskop ermittelt werden. Dazu soll beachtet werden,
  dass $R_i = 50\Omega$ und $\hat{u}_{gen} = 1V$ sind. Die Kapazitätsdekade wird mit den berechneten Werten aus der Vorbereitung eingestellt.
\end{task}
\begin{figure}[h]
  \begin{center}

    \caption{Schaltplan zur Bestimmung der Serienimpedanz eines Drahtwiderstands}
  \end{center}
\end{figure}
\begin{devlist}
  T
  \begin{itemize}
    \item Oszilloskop: Tektronix MDO3012
    \item Funktionsgenerator: Keysight 33210A
    \item HP 4294A Impedance Analyzer
    \item Multimeter: MetraHit X-TRA Multimeter
  \end{itemize}
\end{devlist}

\subsection{Bestimmung von $\hat{u}_C$/$\hat{u}_{gen}$}

\begin{table}[h]
  \begin{center}

    \begin{tabular}{|c|c|c|}
      \hline
          & $500Hz; C=1,01\mu F$ & $1kHz; C= 253,3 nF$ \\
      \hline
      -95 &                      &                     \\
      \hline
      -95 &                      &                     \\
      \hline
      -95 &                      &                     \\
      \hline
      -95 &                      &                     \\
      \hline
      -95 &                      &                     \\
      \hline
      -95 &                      &                     \\
      \hline
      -95 &                      &                     \\
      \hline
      -95 &                      &                     \\
      \hline
      -95 &                      &                     \\
      \hline
      -95 &                      &                     \\
      \hline
      -95 &                      &                     \\
      \hline
      -95 &                      &                     \\
      \hline
      -95 &                      &                     \\
      \hline
    \end{tabular}
    \caption{Messwerte für Versuch 1.1}
    \label{tab:MV}
  \end{center}
\end{table}




\subsection{Auswertung}

Unsere abgelesene Resonanzfrequenz liegt nun also bei: $f_{1Res} = 835kHz$.
Aus dieser können wir mit (\ref{eq:LS}) die gesuchte Serieninduktivität des Drahtwiderstands berechnen: $L_S = 3,633\mu H$\\[3pt]
Um nun die Serienresonanzfrequenz mit der Parallelresonanzfrequenz zu vergleichen, setzen wir unsere errechnete Induktivität in
(\ref{eq:f2}) ein und erhalten dadurch: $f_{2Res} = 648,745kHz$. Wir testen nun mit dem Oszilloskop diese berechnete Resonanzfrequenz, indem wir
die beiden Signale wieder in Phase bringen, wobei wir $f_{2Res} = 652kHz$ ablesen.

Da unsere Rechnungen auf der Annahme beruhen, dass unsere Bauteile, im Besonderen der Kondensator, bei jeder Frequenz die gleichen Kenngrößen haben,
erkennen wir eine mögliche Fehlerquelle. Mit dem HP4294A messen wir bei den bestimmten Resonanzfrequenzen die Kapazität aus und erhalten bei $f_{1Res}$ eine Kapazität
von $C_{f1}=9,53nF$ und bei $f_{2Res}$ eine Kapazität von $C_{f2} = 9,55nF$. Aus den angepassten Werten ergibt sich dann eine Serieninduktivität von $L_S = 3,812\mu F$
und eine Parallelresonanzfrequenz von $f_{2Res} = 666,922kHZ$. Da allerdings diese Messwerte sich implizit auf den "falschen" Messwerten beruhen, sind hier die Toleranzen aufgrund der
Fehlerfortpflanzung größer. Dazu kommen noch Ungenauigkeiten bei den Geräten und deren Bedienung.


\newpage
\section{Resonanz - Schaltverhalten}
\begin{task}
  IIm zweiten Versuch soll durch das Ermitteln des Verlustwiderstands in Abhängigkeit von der Frequenz der Einfluss selbiger auf das Verhalten des Verlustwiderstands erarbeitet werden.
  Gleichzeitig ist das Einstellen einer Kapazitätsdekade auf Resonanzfrequenz mit der Induktivität Gegenstand der Aufgabe.
\end{task}
\begin{figure}[h]
  \begin{center}

    \caption{Schaltplan zur Bestimmung des Verlustwiderstands einer Induktivität}
  \end{center}
\end{figure}
\begin{devlist}
  T
  \begin{itemize}
    \item Oszilloskop: Tektronix MDO3012
    \item Kapazitätsdekade Time Electronics Model 1071 CapBox
    \item Funktionsgenerator: Keysight 33210A
    \item Multimeter: MetraHit X-TRA Multimeter
  \end{itemize}
\end{devlist}
\subsection{Vorbereitung}

Wir wollen zuerst einmal aus den bereits erarbeiten Formeln die Berechnung der Kapazitätsdekade herleiten:
\begin{equation}
  C = \frac{1}{4\cdot\pi^2\cdot f_{Res}^2 \cdot L}
\end{equation}
Aus der Annahme, dass sich bei eingestellter Resonanzfrequenz die Impedanzen der Spule und der Kapazitätsdekade aufheben, können wir die Formel
für die Berechnung des Verlustwiderstands $R_v$ aus dem Spannungsteiler herleiten:
\begin{align*}
  U_2                           & = U_1 \cdot \frac{R_v}{R_v+R}                \\
  \left(R_v + R\right)\cdot U_2 & = U_1 \cdot R_v                              \\
  R_v + R                       & = \frac{U_1}{U_2} \cdot R_v                  \\
  R                             & = \left(\frac{U_1}{U_2} - 1\right) \cdot R_v \\
\end{align*}
Nun teilen wir noch durch $\frac{U_1}{U_2} - 1$ und bekommen:
\begin{equation}
  \frac{R}{\left(\frac{U_1}{U_2} - 1\right)}  = R_v
\end{equation}

Wir messen mit dem MetraHit vorher noch den Gleichstromwiderstand der Spule, um den später bei der Messung zu berücksichtigen.
Dieser beträgt $R_{Cu} = 38,8\Omega$.

\subsection{Versuchsdurchführung}

Wir stellen die Kapazitätsdekade so ein, dass wir bei den Frequenzen $f = 1,2,5,10,20,50,100kHz$ jeweils die Resonanzfrequenz bekommen und Messen dann
$U_1$ und $U_2$ mit dem Oszilloskop. Daraus errechnen wir dann $R_v$ und tragen diese Werte dann in einem Plot ab.

\subsection{Auswertung}
\begin{center}

  \begin{tabular}{|c|c|c|c|c|c|}
    \hline
    $\frac{f}{kHz}$ & $\frac{C_{Rechnung}}{\mu F}$ & $\frac{C_{Messung}}{\mu F}$ & $\frac{U_1}{mV}$ & $\frac{U_2}{mV}$ & $\frac{R_v}{\Omega}$ \\
    \hline
    1               & $1,151377$                   & $1,13019$                   & $952$            & $37,11$          & $40,56$              \\
    \hline
    2               & $0,287$                      & $0,282$                     & $953$            & $37,5$           & $40,96$              \\
    \hline
    5               & $0,046055$                   & $0,04556$                   & $956,1$          & $40,02$          & $43,69$              \\
    \hline
    10              & $0,0151$                     & $0,0113$                    & $952,1$          & $30,93$          & $33,576$             \\
    \hline
    20              & $0,002878$                   & $0,00278$                   & $955$            & $36,1$           & $39,286$             \\
    \hline
    50              & $460pF$                      & $370pF$                     & $959$            & $79$             & $89,772$             \\
    \hline
    100             & $151pF$                      & $30pF$                      & $972$            & $405$            & $714,28$             \\
    \hline
  \end{tabular}
\end{center}
\begin{figure}[h]
  \begin{center}

    \caption{Der Verlustwiderstand doppelt logarithmisch von der Frequenz abgetragen}
  \end{center}
\end{figure}
Aus der Abbildung und den Messwerten folgt nun, dass bei kleinen Frequenzen bis ca. 50kHz die Auswirkungen der Frequenz auf den Verlustwiderstand minimal sind.
Der Verlustwiderstand wird zu einem großen Teil durch den Gleichstromwiderstand $C_{Cu}$ bestimmt. Ab dieser Marke setzt dann der Skin-Effekt ein; hier wird auch die
Frequenzabhängigkeit eines Widerstandes eindrucksvoll klar. Es ist außerdem anzumerken, dass sich daraus, für die Anwendung in echten Schaltkreisen eine Präferenz für niedrige Frequenzen ergibt,
wenn man zum Ziel hat, den Verlustwiderstand einer Induktivität möglichst gering zu halten.
\end{document}