\documentclass{article}

\usepackage{circuitikz}
\usepackage[T1]{fontenc} 
\usepackage[UTF8]{inputenc}
\usepackage{amsmath}
\usepackage{amssymb}
\usepackage{fancyhdr}
\usepackage{graphicx}
\usepackage{hyperref}
\usepackage{tikz}
  \usetikzlibrary{arrows}
  \usetikzlibrary{shapes}
  \usetikzlibrary{arrows.meta,topaths}
  \usetikzlibrary{bending}
  \usetikzlibrary{calc}
\usepackage{anyfontsize}
\usepackage{sectsty}
\usepackage{../assets/scripts/tex/color-env}
\usepackage{anyfontsize}
\usepackage{xcolor}
\definecolor{DarkGreenBlue}{HTML}{264653}
\definecolor{LightGreenBlue}{HTML}{2A9D8F}
\definecolor{LightOrange}{HTML}{E9C46A}
\definecolor{DarkOrange}{HTML}{F4A261}
\definecolor{RedOrange}{HTML}{E76F51}
\definecolor{BrightRed}{HTML}{D62828}
\definecolor{DeepBlue}{HTML}{003049}



\usepackage[ngerman]{babel}
\title{Elektrotechnik 1 - Praktikum 1}


\usepackage[
  includehead,
  headheight = 17mm,
  footskip = \dimexpr\headsep+\ht\strutbox\relax,
  tmargin = 0mm,
  bmargin = \dimexpr17mm+2\ht\strutbox\relax,
]{geometry}





\pagestyle{fancy}
\fancyhead[L]{\leftmark}
\fancyhead[R]{}
\fancyfoot[L]{}
\fancyfoot[C]{\thepage}
\fancyfoot[R]{\includegraphics[scale=0.2]{../assets/images/haw.jpg}}
\renewcommand\headrulewidth{0.5pt}


\begin{document}


\thispagestyle{empty}
\begin{tikzpicture}[remember picture,overlay]

  \fill[DeepBlue] (current page.south west) rectangle (current page.north east);

  \begin{scope}

    \foreach \i in {2.5,...,22}
      {
        \node[rounded corners, DeepBlue!90,draw ,regular polygon, regular polygon sides=6, minimum size=\i cm, ultra thick] at ($(current page.west)+(2.5,-5)$) {} ;
      }

  \end{scope}

  \node[rounded corners,fill=DeepBlue!95,text =DeepBlue!5,regular polygon,regular polygon sides=6, minimum size=2.5 cm,inner sep=0,ultra thick] at ($(current page.west)+(2.5,-5)$) {\LARGE \bfseries 2020};

  \foreach \i in {0.5,...,22}
    {
      \node[rounded corners,DeepBlue!90,draw,regular polygon,regular polygon sides=6, minimum size=\i cm,ultra thick] at ($(current page.north west)+(2.5,0)$) {} ;
    }

  \foreach \i in {0.5,...,22}
    {
      \node[rounded corners,DeepBlue!98,draw,regular polygon,regular polygon sides=6, minimum size=\i cm,ultra thick] at ($(current page.north east)+(0,-9.5)$) {} ;
    }

  \foreach \i in {12}
    {
      \node[fill = DeepBlue,rounded corners,draw=DeepBlue,regular polygon,regular polygon sides=6, minimum size=\i cm,ultra thick] at ($(current page.south east)+(-0.2,-0.45)$) {} ;
    }


  \foreach \i in {21,...,6}
    {
      \node[DeepBlue!95,rounded corners,draw,regular polygon,regular polygon sides=6, minimum size=\i cm,ultra thick] at ($(current page.south east)+(-0.2,-0.45)$) {} ;
    }

  \node[left,DeepBlue!5,minimum width=0.625*\paperwidth,minimum height=3cm, rounded corners] at ($(current page.north east)+(0,-9.5)$){{\fontsize{25}{30} \selectfont \bfseries TITLE OF THE REPORT}};

  \node[left,DeepBlue!10,minimum width=0.625*\paperwidth,minimum height=2cm, rounded corners] at ($(current page.north east)+(0,-11)$){{\huge \textit{Subtitle of the Report}}};

  \node[left,DeepBlue!5,minimum width=0.625*\paperwidth,minimum height=2cm, rounded corners] at ($(current page.north east)+(0,-13)$){{\Large \textsc{Author Name}}};

\end{tikzpicture}

\newpage
\thispagestyle{empty}

\tableofcontents


\newpage

\section{Serieninduktivität eines Drahtwiderstands}
\begin{task}
  IIm ersten Versuch soll die Serieninduktivität eines Drahtwiderstands erst über Messung der Serienresonanzfrequenz bei
  gegebenem Kondensator berechnet werden. Im Anschluss wird dann das Ergebnis genutzt, um das Verhalten der Schaltung bei Parallelresonanzfrequenz zu untersuchen.
\end{task}
\begin{figure}[h]
  \begin{center}

    \caption{Schaltplan zur Bestimmung der Serienimpedanz eines Drahtwiderstands}
  \end{center}
\end{figure}
\begin{devlist}
  T
  \begin{itemize}
    \item Oszilloskop: Tektronix MDO3012
    \item Funktionsgenerator: Keysight 33210A
    \item HP 4294A Impedance Analyzer
    \item Multimeter: MetraHit X-TRA Multimeter
  \end{itemize}
\end{devlist}
\subsection{Vorbereitung}
\begin{figure}

\end{figure}
Zunächst wollen wir die gegebenen Formeln so umstellen, dass wir mit ihnen die gesuchten Größen errechnen können.

Für unsere erste Schaltung gilt:
\begin{equation}
  f_{1Res} = \frac{1}{2\cdot \pi \cdot \sqrt{L_S \cdot C}}
\end{equation}
Durch ein wenig umstellen erhalten wir dann folgende Form, die wir dann quadrieren:
\begin{align*}
  \sqrt{L_S\cdot C} & = \frac{1}{2\cdot\pi \cdot f_{1Res}}    \\
  L_S \cdot C       & = \frac{1}{4\cdot\pi^2\cdot f_{1Res}^2}
\end{align*}
Wir bringen nun zum Schluss noch $C$ auf die andere Seite und erhalten:
\begin{equation}\label{eq:LS}
  L_S = \frac{1}{4\cdot\pi^2\cdot f_{1Res}^2\cdot C}
\end{equation}
Um nun schlussendlich auch die Parallelresonanzfrequenz zu berechnen, gibt es noch eine entsprechende Formel:
\begin{equation}\label{eq:f2}
  f_{2Res} = \frac{\sqrt{1-\frac{R^2}{L_s/C}}}{2\cdot\pi\cdot\sqrt{L_S\cdot C}}
\end{equation}
\subsection{Versuchsdurchführung}
Um die Resonanzfrequenz zu ermitteln, müssen die beiden Spannungen $U_1$ und $U_2$ in Phase sein.
Dazu variieren wir also am Frequenzgenerator die Frequenz des Eingangssignals $U_1$ so lange, bis sich
$U_1$ und $U_2$ am Oszilloskop überlagern. An diesem Punkt ist auch $U_2$ minimal.
Es ist außerdem hilfreich, das Oszilloskop im X-Y-Betrieb laufen zu lassen, damit man die Resonanzfrequenz noch genauer einstellen kann.
Findet sich nämlich in der Form eine genau gerade und scharfe Linie vor, ist die Resonanzfrequenz erreicht.

\subsection{Auswertung}

Unsere abgelesene Resonanzfrequenz liegt nun also bei: $f_{1Res} = 835kHz$.
Aus dieser können wir mit (\ref{eq:LS}) die gesuchte Serieninduktivität des Drahtwiderstands berechnen: $L_S = 3,633\mu H$\\[3pt]
Um nun die Serienresonanzfrequenz mit der Parallelresonanzfrequenz zu vergleichen, setzen wir unsere errechnete Induktivität in
(\ref{eq:f2}) ein und erhalten dadurch: $f_{2Res} = 648,745kHz$. Wir testen nun mit dem Oszilloskop diese berechnete Resonanzfrequenz, indem wir
die beiden Signale wieder in Phase bringen, wobei wir $f_{2Res} = 652kHz$ ablesen.

Da unsere Rechnungen auf der Annahme beruhen, dass unsere Bauteile, im Besonderen der Kondensator, bei jeder Frequenz die gleichen Kenngrößen haben,
erkennen wir eine mögliche Fehlerquelle. Mit dem HP4294A messen wir bei den bestimmten Resonanzfrequenzen die Kapazität aus und erhalten bei $f_{1Res}$ eine Kapazität
von $C_{f1}=9,53nF$ und bei $f_{2Res}$ eine Kapazität von $C_{f2} = 9,55nF$. Aus den angepassten Werten ergibt sich dann eine Serieninduktivität von $L_S = 3,812\mu F$
und eine Parallelresonanzfrequenz von $f_{2Res} = 666,922kHZ$. Da allerdings diese Messwerte sich implizit auf den "falschen" Messwerten beruhen, sind hier die Toleranzen aufgrund der
Fehlerfortpflanzung größer. Dazu kommen noch Ungenauigkeiten bei den Geräten und deren Bedienung.


\newpage
\section{Verlustwiderstand einer Induktivität}
\begin{task}
  IIm zweiten Versuch soll durch das Ermitteln des Verlustwiderstands in Abhängigkeit von der Frequenz der Einfluss selbiger auf das Verhalten des Verlustwiderstands erarbeitet werden.
  Gleichzeitig ist das Einstellen einer Kapazitätsdekade auf Resonanzfrequenz mit der Induktivität Gegenstand der Aufgabe.
\end{task}
\begin{figure}[h]
  \begin{center}

    \caption{Schaltplan zur Bestimmung des Verlustwiderstands einer Induktivität}
  \end{center}
\end{figure}
\begin{devlist}
  T
  \begin{itemize}
    \item Oszilloskop: Tektronix MDO3012
    \item Kapazitätsdekade Time Electronics Model 1071 CapBox
    \item Funktionsgenerator: Keysight 33210A
    \item Multimeter: MetraHit X-TRA Multimeter
  \end{itemize}
\end{devlist}
\subsection{Vorbereitung}

Wir wollen zuerst einmal aus den bereits erarbeiten Formeln die Berechnung der Kapazitätsdekade herleiten:
\begin{equation}
  C = \frac{1}{4\cdot\pi^2\cdot f_{Res}^2 \cdot L}
\end{equation}
Aus der Annahme, dass sich bei eingestellter Resonanzfrequenz die Impedanzen der Spule und der Kapazitätsdekade aufheben, können wir die Formel
für die Berechnung des Verlustwiderstands $R_v$ aus dem Spannungsteiler herleiten:
\begin{align*}
  U_2                           & = U_1 \cdot \frac{R_v}{R_v+R}                \\
  \left(R_v + R\right)\cdot U_2 & = U_1 \cdot R_v                              \\
  R_v + R                       & = \frac{U_1}{U_2} \cdot R_v                  \\
  R                             & = \left(\frac{U_1}{U_2} - 1\right) \cdot R_v \\
\end{align*}
Nun teilen wir noch durch $\frac{U_1}{U_2} - 1$ und bekommen:
\begin{equation}
  \frac{R}{\left(\frac{U_1}{U_2} - 1\right)}  = R_v
\end{equation}

Wir messen mit dem MetraHit vorher noch den Gleichstromwiderstand der Spule, um den später bei der Messung zu berücksichtigen.
Dieser beträgt $R_{Cu} = 38,8\Omega$.

\subsection{Versuchsdurchführung}

Wir stellen die Kapazitätsdekade so ein, dass wir bei den Frequenzen $f = 1,2,5,10,20,50,100kHz$ jeweils die Resonanzfrequenz bekommen und Messen dann
$U_1$ und $U_2$ mit dem Oszilloskop. Daraus errechnen wir dann $R_v$ und tragen diese Werte dann in einem Plot ab.

\subsection{Auswertung}
\begin{center}

  \begin{tabular}{|c|c|c|c|c|c|}
    \hline
    $\frac{f}{kHz}$ & $\frac{C_{Rechnung}}{\mu F}$ & $\frac{C_{Messung}}{\mu F}$ & $\frac{U_1}{mV}$ & $\frac{U_2}{mV}$ & $\frac{R_v}{\Omega}$ \\
    \hline
    1               & $1,151377$                   & $1,13019$                   & $952$            & $37,11$          & $40,56$              \\
    \hline
    2               & $0,287$                      & $0,282$                     & $953$            & $37,5$           & $40,96$              \\
    \hline
    5               & $0,046055$                   & $0,04556$                   & $956,1$          & $40,02$          & $43,69$              \\
    \hline
    10              & $0,0151$                     & $0,0113$                    & $952,1$          & $30,93$          & $33,576$             \\
    \hline
    20              & $0,002878$                   & $0,00278$                   & $955$            & $36,1$           & $39,286$             \\
    \hline
    50              & $460pF$                      & $370pF$                     & $959$            & $79$             & $89,772$             \\
    \hline
    100             & $151pF$                      & $30pF$                      & $972$            & $405$            & $714,28$             \\
    \hline
  \end{tabular}
\end{center}
\begin{figure}[h]
  \begin{center}

    \caption{Der Verlustwiderstand doppelt logarithmisch von der Frequenz abgetragen}
  \end{center}
\end{figure}
Aus der Abbildung und den Messwerten folgt nun, dass bei kleinen Frequenzen bis ca. 50kHz die Auswirkungen der Frequenz auf den Verlustwiderstand minimal sind.
Der Verlustwiderstand wird zu einem großen Teil durch den Gleichstromwiderstand $C_{Cu}$ bestimmt. Ab dieser Marke setzt dann der Skin-Effekt ein; hier wird auch die
Frequenzabhängigkeit eines Widerstandes eindrucksvoll klar. Es ist außerdem anzumerken, dass sich daraus, für die Anwendung in echten Schaltkreisen eine Präferenz für niedrige Frequenzen ergibt,
wenn man zum Ziel hat, den Verlustwiderstand einer Induktivität möglichst gering zu halten.
\newpage

\section{Serienverlustwiderstand eines Becherelkos}
\begin{task}
  IIm dritten und letzten Versuch soll nun der Serienverlustwiderstand eines RC-Glieds in einer Brückenschaltung gemessen und errechnet werden.
  Dazu soll das Variieren einer Widerstands- und Kapazitätsdekade zum Abgleich einer einfachen Brückenschaltung als Methode der Messung geübt werden.
\end{task}
\begin{figure}[h]
  \begin{center}

    \caption{Schaltplan der Brückenschaltung}
  \end{center}
\end{figure}
\begin{devlist}
  T
  \begin{itemize}
    \item Transformator Sennheiser RVZ11
    \item Oszilloskop: Tektronix MDO3012
    \item Kapazitätsdekade: Time Electronics Model 1071 CapBox
    \item Widerstandsdekade: Time Electronics Model 1051 ResBox
    \item Funktionsgenerator: Keysight 33210A
  \end{itemize}
\end{devlist}
\subsection{Vorbereitung}
Unter der Vorraussetzung, dass die Variation des einstellbaren RC-Glieds ein Abgleich der Brückenschaltung hervorruft, gilt:
\begin{align*}
  \frac{C_N}{C_X} & = \frac{R_1}{R_2} \\
  \frac{R_N}{R_X} & = \frac{R_2}{R_1} \\
\end{align*}
Aus diesen Vorraussetzungen folgt dann:
\begin{align}
  \frac{C_N\cdot R_2}{R_1} & = C_X\label{eq:CX} \\
  \frac{R_N\cdot R_1}{R_2} & = R_X\label{eq:RX}
\end{align}
\subsection{Versuchsdurchführung}
Wir stellen am Funktionsgenerator jeweils die Frequenzen $f= 100Hz, 500Hz\text{ und }1kHz$ ein und ermitteln die Kapazität $C_X$ und den Verlustwiderstand $R_X$
und tragen dann die Werte in eine Tabelle ab. Im Anschluss werden die Ergebnisse dann mit dem HP 4294A Impedance Analyzer überprüft.

\subsection{Auswertung}

\begin{center}
  \begin{tabular}{|c|c|c|c|}
    \hline
    $\frac{f}{Hz}$ & $\frac{C_X}{nF}$ & $\frac{R_X}{\Omega}$ & $\frac{U_{Br}}{\mu V}$ \\
    \hline
    100            & $1000$           & $81$                 & $4,2$                  \\
    \hline
    500            & $988$            & $21$                 & $6,4$                  \\
    \hline
    1000           & $898$            & $20$                 & $4,2$                  \\
    \hline
  \end{tabular}
\end{center}
Wir erkennen einen deutlichen Abfall der Kapazität des Becherelkos sowie ihres Serienwiderstandes bei steigender Frequenz. Im
Gegensatz zur Aufgabe 2 ist hier der Einfluss des Skin-Effektes nicht zu verspüren, da hier keine hochfrequenten Signale verwendet werden. Bei der Bestimmung von unbekannten
passiven Bauelementen ist es also förderlich auch die Frequenz bei der Analyse der Ergebnisse mit einzubeziehen und dann gegebenenfalls auch eine Bezugsfrequenz festzulegen.

\end{document}