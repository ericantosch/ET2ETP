\documentclass{article}

\usepackage{circuitikz}
\usepackage[T1]{fontenc} 
\usepackage[UTF8]{inputenc}
\usepackage{amsmath}
\usepackage{amssymb}
\usepackage{fancyhdr}
\usepackage{graphicx}
\usepackage{hyperref}
\usepackage{tikz}
  \usetikzlibrary{arrows}
  \usetikzlibrary{shapes}
  \usetikzlibrary{arrows.meta,topaths}
  \usetikzlibrary{bending}
  \usetikzlibrary{calc}
\usepackage{anyfontsize}
\usepackage{sectsty}
\usepackage{../assets/scripts/tex/color-env}
\usepackage{anyfontsize}
\usepackage{xcolor}
\definecolor{DarkGreenBlue}{HTML}{264653}
\definecolor{LightGreenBlue}{HTML}{2A9D8F}
\definecolor{LightOrange}{HTML}{E9C46A}
\definecolor{DarkOrange}{HTML}{F4A261}
\definecolor{RedOrange}{HTML}{E76F51}
\definecolor{BrightRed}{HTML}{D62828}
\definecolor{DeepBlue}{HTML}{003049}



\usepackage[ngerman]{babel}
\title{Elektrotechnik 1 - Praktikum 3}


\usepackage[
  includehead,
  headheight = 17mm,
  footskip = \dimexpr\headsep+\ht\strutbox\relax,
  tmargin = 0mm,
  bmargin = \dimexpr17mm+2\ht\strutbox\relax,
]{geometry}





\pagestyle{fancy}
\fancyhead[L]{\leftmark}
\fancyhead[R]{}
\fancyfoot[L]{}
\fancyfoot[C]{\thepage}
\fancyfoot[R]{\includegraphics[scale=0.2]{../assets/images/haw.jpg}}
\renewcommand\headrulewidth{0.5pt}

\begin{document}

\thispagestyle{empty}
\begin{tikzpicture}[remember picture,overlay]

  \fill[DeepBlue] (current page.south west) rectangle (current page.north east);

  \begin{scope}

    \foreach \i in {2.5,...,22}
      {
        \node[rounded corners, DeepBlue!90,draw ,regular polygon, regular polygon sides=6, minimum size=\i cm, ultra thick] at ($(current page.west)+(2.5,-5)$) {} ;
      }

  \end{scope}

  \node[rounded corners,fill=DeepBlue!95,text =DeepBlue!5,regular polygon,regular polygon sides=6, minimum size=2.5 cm,inner sep=0,ultra thick] at ($(current page.west)+(2.5,-5)$) {\LARGE \bfseries 2020};

  \foreach \i in {0.5,...,22}
    {
      \node[rounded corners,DeepBlue!90,draw,regular polygon,regular polygon sides=6, minimum size=\i cm,ultra thick] at ($(current page.north west)+(2.5,0)$) {} ;
    }

  \foreach \i in {0.5,...,22}
    {
      \node[rounded corners,DeepBlue!98,draw,regular polygon,regular polygon sides=6, minimum size=\i cm,ultra thick] at ($(current page.north east)+(0,-9.5)$) {} ;
    }

  \foreach \i in {12}
    {
      \node[fill = DeepBlue,rounded corners,draw=DeepBlue,regular polygon,regular polygon sides=6, minimum size=\i cm,ultra thick] at ($(current page.south east)+(-0.2,-0.45)$) {} ;
    }


  \foreach \i in {21,...,6}
    {
      \node[DeepBlue!95,rounded corners,draw,regular polygon,regular polygon sides=6, minimum size=\i cm,ultra thick] at ($(current page.south east)+(-0.2,-0.45)$) {} ;
    }

  \node[left,DeepBlue!5,minimum width=0.625*\paperwidth,minimum height=3cm, rounded corners] at ($(current page.north east)+(0,-9.5)$){{\fontsize{25}{30} \selectfont \bfseries ET2 - Praktikum 6}};

  \node[left,DeepBlue!10,minimum width=0.625*\paperwidth,minimum height=2cm, rounded corners] at ($(current page.north east)+(0,-11)$){{\huge \textit{Transiente Vorgänge}}};

  \node[left,DeepBlue!5,minimum width=0.625*\paperwidth,minimum height=2cm, rounded corners] at ($(current page.north east)+(0,-13)$){{\Large \textsc{Florian Tietjen\hspace{0.5cm}Eric Antosch}}};

\end{tikzpicture}

\newpage
\thispagestyle{empty}

\tableofcontents


\newpage

\section{Vorbereitung 2 - Transiente Vorgänge am Kondensatornetzwerk}

\begin{figure}[h]
  \begin{center}
    \includegraphics[scale=1]{../assets/images/ET2P5/vorbereitung 2.jpg}
    \caption{Versuchsaufbau des Netzwerkes}
  \end{center}
\end{figure}

Zunächst soll die Anfangsspannung am Kondensator $u_C$ (0) bestimmt werden:

\begin{align*}
  u_C(0) &= U_{R1} = \frac{R1}{R1+R3} \cdot U1 \\
  u_C(0) &= \frac{100\Omega}{100\Omega + 50\Omega} \cdot 30V = 20V
\end{align*}

Nun wird die Spannung am Kondensator für den eingeschwungenen Zustand $u_C$ (t) für $t \rightarrow \infty$ berechnet:
\begin{align*}
  u_C(t)&= U_0 \cdot e^{-\frac{t}{RC}} = 0V
\end{align*}

Bestimmung der Zeitkonstante $\tau$ für geschlossenen Schalter:
\begin{align*}
  \tau &= RC = ((R3 // R1) + R2) \cdot C = 33\mu s
\end{align*}

Bei geöffnetem Schalter:
\begin{align*}
  \tau &= RC = (R1 + R2) \cdot C = 300\mu s
\end{align*}

Daraus ergeben sich die allgemeinen Formeln für den zeitlichen Verlauf der Spannung und des Stroms für alle t >= $100\mu s$

\begin{align*}
  u_C(t)&= 20V \cdot e^{-\frac{t-100\mu s}{300\mu s}}\\
  i_C(t)&= -66,7mA \cdot e^{-\frac{t-100\mu s}{300\mu s}}
\end{align*}

\newpage

\section{Vorbereitung 3 - Transiente Vorgänge am Spulennetzwerk}

\begin{figure}[h]
  \begin{center}
    \includegraphics[scale=1]{../assets/images/ET2P5/vorbereitung 3.jpg}
    \caption{Versuchsaufbau des Netzwerkes}
  \end{center}
\end{figure}

Die folgende Schaltung ist das Ersatzschaltbild für $t0 < t < t1$:
\begin{figure}[h]
  \begin{center}
    \includegraphics[scale=0.3]{../assets/images/ET2P5/t0 t t1.JPG}
    \caption{ESB für $t0 < t < t1$ }
  \end{center}
\end{figure}

Die folgende Schaltung ist das Ersatzschaltbild für $t1 < t$:
\begin{figure}[h]
  \begin{center}
    \includegraphics[scale=0.3]{../assets/images/ET2P5/t1 kleiner t.JPG}
    \caption{ESB für $t1 < t$ }
  \end{center}
\end{figure}
\newpage

Berechnung des Stroms $i_{L1}$ für $t_0$:
\begin{align*}
  i_{L1} = 0A
\end{align*}
Da die Spule anfangs wie ein unendlicher hoher Widerstand wirkt, ist der Strom bei $t_0$ 0A.\\


Für $i_{L1}$, wenn $t \rightarrow \infty$ und S2 stets geöffnet ist, gilt:
\begin{align*}
  i_{L1}&= i_{R_g} = \frac{U1}{R_g} = \frac{6V}{25\Omega}\\
  i_{L1}&= 240mA
\end{align*}
Dabei wurde die Spule wie kurzgeschlossen betrachtet. \\

Die Zeitkonstanten für $t_0 < t < t_1$ ergibt sich aus:
\begin{align*}
  \tau_{t0} &= \frac{L1}{R} = \frac{1mH}{25\Omega}=40\mu s
\end{align*}

Gleichung für den Spulenstrom $i_L$ lautet:
\begin{align*}
  i_{L1}(t) = 240mA (1 - e^{-\frac{t}{40\mu s}})\\
  i_{L1}(t=t_1=100\mu) = 220mA
\end{align*}

Für die Zeitkonstante sowie für den Strom wenn beide Schalter geschlossen sind für $t \rightarrow \infty$ gilt:
\begin{align*}
  \tau = \frac{L}{R} = \frac{L}{(R1//R3)+R2} = \frac{1mH}{9\Omega} = 111,1\mu s\\
  t\rightarrow\infty : i_L(t) = 133,33mA
\end{align*}

Da sich die Spule ab t = $100\mu s$ wieder entlädt gilt:
\begin{align*}
  i_{L1}(t1=100\mu) - i_{L1}(t=t\rightarrow \infty) &= 220mA - 133,33mA = 86,67mA\\
  i_{L1}(t) &= 133,33mA + 86,67mA\cdot e^{-\frac{t-100\mu s}{111,1\mu s}})
\end{align*}


Lösen des Stroms nach $200\mu$s:
\begin{align*}
  i_{L1}(t=200\mu s) &= 133,33mA + 86,67mA\cdot e^{-\frac{t-100\mu s}{111,1\mu s}})
  i_{L1}(200\mu s) &= 168,69mA
\end{align*}


\end{document}
