\documentclass{scrarticle}

\usepackage{circuitikz} %Für die Schaltpläne
\usepackage[T1]{fontenc} 
\usepackage[UTF8]{inputenc}
\usepackage{amsmath}
\usepackage{amssymb}
\usepackage{fancyhdr}
\usepackage{graphicx}
\usepackage{tikz}
\usepackage[ngerman]{babel}
    \usetikzlibrary{arrows}
    \usetikzlibrary{arrows.meta,topaths}
    \usetikzlibrary{bending}
\title{Elektrotechnik 1 - Praktikum 1}
\subtitle{Gleichstromanalyse mit LTSpice}
\author{Florian Tietjen \and Eric Antosch}

%\usepackage{scrlayer-scrpage}

\usepackage[
  includehead,
  headheight = 17mm,
  footskip = \dimexpr\headsep+\ht\strutbox\relax,
  tmargin = 0mm,
  bmargin = \dimexpr17mm+2\ht\strutbox\relax,
]{geometry}





\usepackage{amsthm}
\newtheorem{lemma}{Lemma}

\renewcommand\qedsymbol{$\square$}
\pagestyle{fancy}
\fancyhead[L]{\leftmark}
\fancyhead[R]{}
\fancyfoot[L]{}
\fancyfoot[C]{\thepage}
\fancyfoot[R]{\includegraphics[scale=0.2]{haw.jpg}}
\renewcommand\headrulewidth{0.5pt}


\begin{document}

\maketitle
\thispagestyle{empty}

\newpage

\tableofcontents
\thispagestyle{empty}

\newpage

\section{Einführung}

Der Gegenstand unserer Untersuchungen entspricht auf dem Papier der Änderung eines
Lastwiderstandes während einer bestimmten Zeit und dessen Auswirkung auf den Rest der Schaltung (z.B. eine Brückenschaltung).
Wheatstonesche Brücken werden jedoch vermehrt in der Messtechnik eingesetzt, um bestimmte Werte messbar bzw. messbarer zu machen.

Dabei wird die Brücke über die anderen Widerstände abgeglichen, d.h. ein Messgerät, was an den Punkten A und B (siehe Abbildung unten)
angeschlossen werden würde, würde keine Potentialunterschiede und damit keine Spannung messen. Es kommt bei diesem Verfahren zu weniger Fehlern,
welche in manchen Geräten von sehr mächtiger Konsequenz sein können. Es ist daher ratsam, sich mit dem Einfluss von bestimmten Kenngrößen auf eine solche
Art von Schaltung eingehend Gedanken zu machen.


\section{Belastete Brückenschaltung an einer Spannungsquelle}

\begin{center}
  \begin{circuitikz}[european]
  \tikzstyle{obj}  = [circle, minimum width=5pt, fill, inner sep=0pt]
  \draw (0,0) to[V<=$U_0$, i_=$I_0$] (0,6) -- 
  (2,6) to[R, v=$U_1$, l=$R_1$, i^=$I_1$, *-] (2,4) -* (2,2)
  to[R=$R_2$, v=$U_2$, i^=$I_2$, *-*] (2,0) -- (0,0);
  \draw (2,6) -- (5,6) to[R=$R_3$, i^=$I_3$, v=$U_3$, -*] (5,4) -- (7,4)
   to[R=$R_L$, i_=$I_L$, fill=red!70!white] (7,2) -- (2,2);
  \draw (5,4) -- (5,2) to[R=$R_4$, v=$U_4$, i^=$I_4$] (5,0) -- (2,0);
  \draw[-latex] (6,3.5) -- (6,2.5);
  \draw (6,3) node[anchor=east] {$U_L$};
  \node [obj, label=above:A] (xa) at (6,4) {};
  \node [obj, label=below:B] (za) at (6,2) {};
\end{circuitikz}
\end{center}

\subsection{Vorbereitung}
\begin{abstract}
  \textbf{Aufgabe 1.1.1} Bestimmen Sie eine allgemeine Formel für die Berechnung der Ersatzspannungsquelle der Brückenschaltung,
  die in Abschnitt 2.1 angegeben ist, mit beliebigen $R_1, R_2, R_3, R_4$, wobei der Lastwiderstand $R_L$ nicht angeschlossen ist
  und geben Sie $U_q, I_k, R_i$ mit dieser Methode an.
\end{abstract}
Um hier auf die richtigen Lösung zu kommen, müssen wir die Potentiale $\phi_1$ und $\phi_2$ bestimmen, die an den Punkten A und B anliegen.
Hier können wir die Spannungsteilerregel anwenden, um auf den Wert $U_q$ zu kommen.
\begin{equation*}
  \begin{aligned}
    U_A &= U_0 \cdot \frac{R_3}{R_3+R_4}\\
    U_B &= U_0 \cdot \frac{R_1}{R_1+R_2}
  \end{aligned}
\end{equation*}
\begin{equation}\label{eq:Uq}
  U_q = U_{B-A} = U_0 \cdot \left(\frac{R_3}{R_3+R_4} - \frac{R_1}{R_1+R_2}\right)
\end{equation}
Für den Innenwiderstand der Ersatzspannungsquelle ergibt sich jetzt (siehe Anhang für das dafür genutzte Schaltbild):
\begin{equation}\label{eq:rg}
  \frac{1}{R_i} = \left(\frac{1}{R_1} + \frac{1}{R_2}\right) + \left(\frac{1}{R_3} + \frac{1}{R_4}\right)
\end{equation}
Durch die theoretischen Werte, die wir mit (\ref{eq:Uq}) und (\ref{eq:rg}) berechnet haben, können wir nun auch den fehlenden Wert $I_k$ berechnen:
\begin{equation}
  \label{eq:Ik}
  I_k = \frac{U_q}{R_i}
\end{equation}
\begin{abstract}
  \textbf{Aufgabe 1.1.2} Stellen Sie nun dar, was passieren würde, wenn alle $R_1 = R_3$ und $R_2 = R_4$
  gelten würde und was das für die Brückenschaltung bedeutet.
\end{abstract}
Da nun die oben genannten jBedingungen gelten, folgt, unabhängig von den Werten der Widerstände aus (\ref{eq:Uq}), dass 
die Quellenspannung der Ersatzspannungsquelle nun $0V$ beträgt. Man sagt auch, die Brückenschaltung ist \textit{abgeglichen}. Demnach ist nun auch der Wert für $I_k = 0A$. Allein der Innenwiderstand der 
Ersatzspannungsquelle lautet nun:
\begin{equation*}
  \frac{1}{R_i} = \left(\frac{1}{R_1} + \frac{1}{R_2}\right) + \left(\frac{1}{R_3} + \frac{1}{R_4}\right) = 2\cdot\left(\frac{1}{R_1} + \frac{1}{R_2}\right)
\end{equation*}
\begin{abstract}
  \textbf{Aufgabe 1.1.3} Berechnen Sie nun den Wert der Quellenspannung $U_q$, des Innenwiderstands $R_i$ und des Kurzschlussstromes $I_k$ der Ersatzspannungsquelle.
\end{abstract}
Mit den oben gezeigten Formeln aus \textbf{Aufgabe 1.1.1} findet sich nun die Werte mit:
\begin{equation*}
  \begin{aligned}
    U_q &= U_0 \cdot \left(\frac{R_3}{R_3+R_4} - \frac{R_1}{R_1+R_2}\right) \\
        &= 10V \cdot \left(\frac{10k\Omega}{10k\Omega+47k\Omega} - \frac{22k\Omega}{22k\Omega+33k\Omega}\right) = 2,2456V.\\
    R_i &= \left(\frac{1}{R_i}\right)^{-1} = \left(\left(\frac{1}{R_1} + \frac{1}{R_2}\right) + \left(\frac{1}{R_3} + \frac{1}{R_4}\right)\right)^{-1} \\
        &= \left(\left(\frac{1}{22k\Omega} + \frac{1}{33k\Omega}\right) + \left(\frac{1}{10k\Omega} + \frac{1}{47k\Omega}\right)\right)^{-1} = 5,075k\Omega\\
    I_k &= \frac{U_q}{R_i} = \frac{2,2456V}{5,075k\Omega} = 4,4246\cdot 10^{-4}A
  \end{aligned}
\end{equation*}
\newpage

\subsection{Simulation}


\subsubsection{Durchführung}
Wir simulieren nun in LTSpice mittels einer zeitabhängigen Funktion des $R_L$ (in der Grafik hellrot makiert) den Einfluss des Lastwiderstandes
auf die verschiedenen Werte der abgeglichnen belasteten Brückenschaltung und schauen uns zudem den Lastleistungswert an. 
Um diese Begebenheit in LTSpice zu simulieren, haben wir den Widerstandswert des Lastwiderstandes als Funktion der Zeit mit dem 
Proportionalitätsfaktor $2200$ erstellt und dann von $t_0 = 0s$ bis $t_{end} = 1000s$ die Wertänderung betrachtet. Die Werte werden dann über
sogenannte Traces von LTSpice auf dem Bildschirm ausgegeben. Diese werden dann als .txt-Tabelle exportiert und dann als Graph dargestellt (siehe Abbildung 1).
\subsubsection{Beobachtung}
Die Werte für die Spannung sind in Rot in Abbildung 1 auf der nächsten Seite dargestellt. Man erkennt hier, dass, mit steigendem Widerstandswert,
auch die Spannung steigt. Jedoch ist der Verlauf nicht linear sondern nähert sich asymptotisch der Quellenspannung der Ersatzspannungsquelle
$U_q = 2,2456V$.
Typisch für die Kurve eines Leistungswerts am veränderlichen Lastwiderstand ist dieser an einem bestimmten Punkt am höchsten, sinkt danach aber wieder; wir bezeichnen
diesen Punkt im Folgenden mit $P_{max}$.
\begin{figure}[h]
  \begin{center}
    \includegraphics[scale=0.5]{pw_volt.png}
    \caption{Die Graphen der Leistung und Spannung in Abhängigkeit von dem Lastwiderstand $R_L$}
  \end{center}
\end{figure}
Der Punkt $P_{max}$ ist in diesem Fall der auf der Abbildung 1 erkennbare Hochpunkt der Funktion für die Lastleistung.
\begin{equation}\label{eq:pmax}
  \psi_n(\lambda, \mu) = \lambda \cdot \mu \land \lambda + \mu = x \in \mathbb{R} \text{ für beliebiges }n\in\mathbb{N}
\end{equation}
In der Aussage (\ref{eq:pmax}) definieren wir die Funktion der Leistung in Abhängigkeit von Strom und Spannung am Lastwiderstand. Wir überlegen uns, dass das Maximum der Leistung erreicht ist, wenn die Differenz der Stromstärke und der Spannung Null ist, sie also gleich sind (Der Beweis findet sich im Anhang).
Es ist außerdem anzumerken, dass sich der höchste Leistungswert an dem Punkt findet, an dem der Innenwiderstand der Ersatzspannungsquelle angesiedelt ist. Grafisch lässt sich dieser Punkt als Schnittpunkt der Graphen für Strom und Spannung interpretieren (siehe Abbildung 2).
\begin{figure}[h]
  \begin{center}
    \label{fig:all}
    \includegraphics[scale=0.5]{pw_volt_rsv2.png}
    \caption{Die Graphen für die Spannung und den Strom treffen sich genau an dem Punkt, an welchem die Leistung maximal ist.}
  \end{center}
\end{figure}
Setzen wir nun alle anderen Werte als Funktion vom Widerstand, sondern betrachten den Wert der Stromstärke
als Funktion der Spannung, so bekommen wir eine wohlbekannte Generatorkennlinie, die ihren Nullpunkt auf dem Wert $U_q = 2,2456V$ und Y-Achsenabschnitt auf den Wert $I_k = 4,4246\cdot 10^{-4}$ hat und somit der Kennlinie unserer
Ersatzspannungsquelle entspricht (siehe Abbildung 3). Dieser Zusammenhang stellt wiederum nur erneut die Begebenheiten des Versuches dar,
da wir mit dem Kurzschlussstrom denjenigen Strom mit $\lim_{R_L\to 0} \frac{U_q}{R_G} = I_k$ und mit der Leerlaufspannung diejenige Spannung $\lim_{R_L\to\infty} I_k\cdot R_G = U_q$ bezeichnen, wobei im Falle der Ersatzspannungsquelle
$R_G = R_i + R_L$ gilt.
\begin{figure}[h]
  \begin{center}
    \includegraphics[scale=0.5]{curr_volt.png}
    \caption{Generatorkennlinie der Ersatzspannungsquelle}
  \end{center}
\end{figure}
\newpage
\section{Belastete Brückenschaltung an einer Stromquelle}
\begin{center}
  \begin{circuitikz}[european]
  \tikzstyle{obj}  = [circle, minimum width=5pt, fill, inner sep=0pt]
  \draw (0,0) to[isource, i_=$I_0$, l=$I_0$] (0,6) -- 
  (2,6) to[R, v=$U_1$, l=$R_1$, i^=$I_1$, *-] (2,4) -* (2,2)
  to[R=$R_2$, v=$U_2$, i^=$I_2$, *-*] (2,0) -- (0,0);
  \draw (2,6) -- (5,6) to[R=$R_3$, i^=$I_3$, v=$U_3$, -*] (5,4) -- (7,4)
   to[R=$R_L$, i_=$I_L$, fill=red!70!white] (7,2) -- (2,2);
  \draw (5,4) -- (5,2) to[R=$R_4$, v=$U_4$, i^=$I_4$] (5,0) -- (2,0);
  \draw[-latex] (6,3.5) -- (6,2.5);
  \draw (6,3) node[anchor=east] {$U_L$};
  \node [obj, label=above:A] (xa) at (6,4) {};
  \node [obj, label=below:B] (za) at (6,2) {};
\end{circuitikz}
\end{center}
\subsection{Vorbereitung}
\begin{abstract}
  \textbf{Aufgabe 1.2.1} Ermitteln Sie mithilfe der obenstehenden Zeichnung die Werte $R_i, U_q, I_k$ für die Ersatzspannungsquelle
  und geben Sie die allgemeinen Formeln für die Werte mit beliebigen $R_{1...4}$ an.
\end{abstract}
Um die Aufgabe so analog wie möglich zu gestalten, berechnen wir zuerst einmal den Gesamtwiderstand der vorliegenden Schaltung. Dieser wird berechnet durch:
\begin{equation}\label{eq:RG}
  \frac{1}{R_G} = \frac{1}{R_1 + R_2} + \frac{1}{R_3 + R_4}
\end{equation}
Zusammen mit dem Gesamtstrom $I_G$, der uns angegeben ist, können wir mit $R_G \cdot I_G = U_G$ berechnen. Ab hier können wir uns nun an der
\textbf{Aufgabe 1.1.1} orientieren und mit (\ref{eq:Uq}) den Wert $U_q$ erhalten. Nur der Innenwiderstand der Ersatzspannungsquelle unterscheidet sich noch
von dem vorherigen Wert, da die Stromquelle als Unterbrechung in der Leitung und nicht als Kurzschluss gewertet wird (siehe Anhang für entsprechende Zeichnung).
\begin{equation}
  \label{eq:RI}
  \frac{1}{R_i} = \frac{1}{R_1 + R_3} + \frac{1}{R_2 + R_4}
\end{equation}
mit (\ref{eq:Ik}) können wir nun auch endlich den fehlenden Wert berechnen.
\begin{abstract}
  \textbf{Aufgabe 1.2.2} Berechnen Sie die Werte mithilfe der Formeln, die sie gerade erarbeitet haben und den vorgebenen Werten aus der Aufgabenstellung.
\end{abstract}
Wir rechnen:
\begin{equation*}
  \begin{aligned}
    \left(\frac{1}{R_G}\right)^{-1} &= \left(\frac{1}{R_1 + R_2} + \frac{1}{R_3 + R_4}\right)^{-1} = \left(\frac{1}{55k\Omega} + \frac{1}{57k\Omega}\right)^{-1} = 27,991k\Omega\\
    U_G &= I_G \cdot R_G = 1mA \cdot 27,991k\Omega = 27,991V\\
  \end{aligned}
\end{equation*}
Nun können wir mit $U_G$ wie in \textbf{Aufgabe 1.1.3} rechnen und erhalten mit (\ref{eq:Uq}), (\ref{eq:Ik}) und (\ref{eq:RI}) folgende Werte:
\begin{equation*}
  \begin{aligned}
U_q &= U_G \cdot \left(\frac{R_3}{R_3+R_4} - \frac{R_1}{R_1+R_2}\right) \\
    &= 27,991V \cdot \left(\frac{10k\Omega}{10k\Omega+47k\Omega} - \frac{22k\Omega}{22k\Omega+33k\Omega}\right) = 6,2857V.\\
R_i &= \left(\frac{1}{R_i}\right)^{-1} = \left(\left(\frac{1}{R_1} + \frac{1}{R_2}\right) + \left(\frac{1}{R_3} + \frac{1}{R_4}\right)\right)^{-1} \\
    &= \left(\left(\frac{1}{22k\Omega} + \frac{1}{33k\Omega}\right) + \left(\frac{1}{10k\Omega} + \frac{1}{47k\Omega}\right)\right)^{-1} = 22,857k\Omega\\
I_k &= \frac{U_q}{R_i} = \frac{6,2857V}{22,857k\Omega} = 2,75\cdot 10^{-4}A
  \end{aligned}
\end{equation*}
\begin{figure}[h]
  \begin{center}
  \includegraphics[scale=0.5]{pw_volt2.png}
  \caption{Die Leistung und Spannung des zweiten Versuches}
  \end{center}
\end{figure}
\newpage
\subsection{Simulation}
\subsubsection{Durchführung}
Wir verwenden das gleiche Setup wie in der Durchführung des ersten Versuches, nur, dass wir hier eine Stromquelle verwenden.
Alle Werte werden wie gewohnt durch die Simulation errechnet und auf Graphen eingezeichnet.
\subsubsection{Beobachtung}

Auch hier ergibt sich ein ähnliches Schema, wie in Aufgabe 1. Sowohl Spannung als auch Strom nähern sich asymptotisch einem bestimmten Wert an ($\text{von }I_k\text{ für I, bis }U_q\text{ für U}$).
Analog zur gemachten Überlegung in (\ref{eq:pmax}) ist auch hier das Maximum der Leistungsfunktion bei dem Schnittpunkt der beiden Graphen zu sehen.
\begin{figure}[h]
  \begin{center}
\includegraphics[scale=0.5]{pw_volt_rsjp2.png}
\caption{Der Schnittpunkt der beiden Werte und erneut der Hochpunkt der Leistungskurve beim Schnittpunkt}
  \end{center}
\end{figure}
\begin{figure}
  \begin{center}
    \includegraphics[scale=0.5]{curr_volt2.png}
    \caption{Die Generatorkennlinie des zweiten Versuchs}
  \end{center}
\end{figure}


\section{Gleichstromschaltung mit mehreren Quellen}
\begin{center}
  

\begin{circuitikz}[european]
  \tikzstyle{obj}  = [circle, minimum width=5pt, fill, inner sep=0pt]
  \draw (0,0) to[isource, i<=$I_1$, l=$I_q$] (0,5) -- (5,5) to[R=$R_3$, v=$U_3$] (7,5) -- (8,5) to[R=$R_4$, v=$U_4$] (8, 3) to[V=$U_{q1}$] (8,0) -- (0,0);
  \draw (2,0) to[short, *-] (2,1) to[R=$R_1$,v<=$U_1$] (2,4) to[short, -*] (2,5);
  \draw (4,0) to[V<=$U_{q0}$, *-] (4,2) to[R=$R_2$,v=$U_2$, -*] (4,5);
  \node[obj, label=above:A] (xa) at (7,5) {};
  \node[obj, label=below:B] (xa) at (7,0) {};
  \node[anchor=east] (Ua) at (7,2.5) {$U_{ab}$};
  \draw[-latex] (7, 3.5) -- (7, 1.5);
  \draw (4,0) to (4,-0.5) node[ground]{};
\end{circuitikz}
\end{center}
\subsection{Vorbereitung}
\begin{abstract}
  \textbf{Aufgabe 1.3.1} Berechnen Sie die Spannung $U_{ab}$ mit den Werten, die in der Aufgabenstellung gegeben sind.
\end{abstract}
Um einen möglichst effizienten Rechenverlauf zu erreichen, wandeln wir zuerst die Stromquelle $I_q$ zu einer Spannungsquelle um. Dabei
wird der Widerstand $R_1$ zum Innenwiderstand $R_i$ unserer neuen Spannungsquelle. Um dann den Spannungswert der Quelle zu berechnen, rechnen wir:
\begin{equation*}
  U_0 = I_q \cdot R_i = 2A \cdot 8\Omega = 16V 
\end{equation*}
Zu beachten ist zudem, dass sich die Richtung der Spannungsquelle in Relation zu ihrer Stromquelle umdreht.
\begin{center}
  \begin{circuitikz}[european]
    \tikzstyle{obj}  = [circle, minimum width=5pt, fill, inner sep=0pt]
    \draw (0,0) to[V=$U_{q2}$] (0,2) to[R=$R_1$, v<=$U_1$] (0,4) -- (0,5) -- (5,5) to[R=$R_3$, v=$U_3$] (7,5) -- (8,5) to[R=$R_4$, v=$U_4$] (8, 3) to[V=$U_{q1}$] (8,0) -- (0,0);
    \draw (4,0) to[V<=$U_{q0}$, *-] (4,2) to[R=$R_2$,v=$U_2$, -*] (4,5);
    \node[obj, label=above:A] (xa) at (7,5) {};
    \node[obj, label=below:B] (xa) at (7,0) {};
    \draw[thin, <-, >=triangle 45] (2.2,2.5)node{$i_1$}  ++(-160:1) arc (-160:170:1);
    \draw[thin, <-, >=triangle 45] (6,2.5)node{$i_2$}  ++(-160:1) arc (-160:170:1);
    \draw (4,0) to (4,-0.5) node[ground]{};
  \end{circuitikz}
\end{center}
Wir erkennen nun zwei Maschenumläufe, die das Berechnen der gesuchten Ströme denkbar einfach machen. Wir stellen also unsere Gleichungen auf mit:
\begin{equation*}
  \begin{aligned}
    K1: 0 &= I_1 - I_2 + I_3\\ 
    M1: 0 &= U_{q2} + U_{q0} - R_1\cdot I_1 - R_2\cdot I_2\\
    M2: 0 &= -U_{q0} + U_{q1} + R_2 \cdot I_2 + (R_3 + R_4) \cdot I_3
  \end{aligned}
\end{equation*}
Mit drei Unbekannten mit jeweils drei unabhängigen Gleichungen stellen wir nun die erweiterte Koeffizientenmatrix auf und erhalten:
\begin{equation*}
  \left(
  \begin{matrix}
    1 & -1 & 1\\
    R_1 & R_2 & 0 \\
    0 & R_2 & R_3 + R_4 \\
  \end{matrix}
  \left|
  \begin{matrix}
    0 \\ U_{q2} + U_{q0} \\ - U_{q1} + U_{q0}\\
  \end{matrix}
  \right)\right.
\end{equation*}
Diese lösen wir nun mithilfe des Gauss-Verfahren und bekommen nun die Werte $I_1 = 3,5057A, I_2 = 1,095A, I_3 = -2,41A$.

Mithilfe dieser Werte können wir nun die Spannungsabfälle entlang der Widerstände berechnen. Der Spannungsabfall von $R_2$
ist nun $U_2 = I_2 \cdot R_2 = 1,0954A \cdot 10\Omega = 10,954V$, womit das Potential des nördlichen Knoten bei $\phi_0 = U_{q0} - U_2 = 23V - 10,954V = 12,05V$ liegt. 
Berechnen wir nun noch den Spannungspfeil über dem Widerstand $R_3$, so erhalten wir den Wert $U_3 = I_3 \cdot R_3 = -2,41A \cdot 1\Omega = -2,41V$. 
Damit findet sich das Potential am Punkt A mit $\phi_a = \phi_0 - U_3 = 10,954V - (-2.41V) = 14,4V$. Weil der Punkt B am Erdpotential liegt, ist die Spannung zwischen Punkt A und B somit gleich dem Potential am Punkt A, also $U_{ab} = 14,4V$.
\newpage
\begin{abstract}
  \textbf{Aufgabe 1.3.2} Berechnen Sie nun im Bezug auf die linke Seite der Schaltung die Werte $U_q, I_k, R_i$ einer Ersatzspannungsquelle/Ersatzstromquelle, wobei die Punkte A und B die Klemmen der Quelle darstellen.
\end{abstract}
Unsere neue Schaltung sieht nun folgendermaßen aus:
\begin{center}
  \begin{circuitikz}[european]
    \tikzstyle{obj}  = [circle, minimum width=5pt, fill, inner sep=0pt]
    \draw (0,0) to[isource, i<=$I_1$, l=$I_q$] (0,5) -- (5,5) to[R=$R_3$, v=$U_3$] (7,5);
    \draw (2,0) to[short, *-] (2,1) to[R=$R_1$,v<=$U_1$] (2,4) to[short, -*] (2,5);
    \draw (0,0) -- (7,0);
    \draw (4,0) to[V<=$U_{q0}$, *-] (4,2) to[R=$R_2$,v=$U_2$, -*] (4,5);
    \node[obj, label=above:A] (xa) at (7,5) {};
    \node[obj, label=below:B] (xa) at (7,0) {};
    \node[anchor=east] (Ua) at (7,2.5) {$U_{ab}$};
    \draw[-latex] (7, 3.5) -- (7, 1.5);
    \draw (4,0) to (4,-0.5) node[ground]{};
  \end{circuitikz}
\end{center}
Um nun die weiteren Werte zu berechnen, transformieren wir unsere Spannungsquelle $U_{q0}$ in eine Stromquelle um. Dabei gilt:
\begin{equation*}
  I_q = \frac{U_{q0}}{R_2} = 2,3A
\end{equation*}
Hierbei ist erneut zu beachten, dass wir die Richtung unserer neuen Stromquelle ändern müssen.
\begin{center}
  \begin{circuitikz}[european]
    \tikzstyle{obj}  = [circle, minimum width=5pt, fill, inner sep=0pt]
    \draw (0,0) to[isource, i<=$I_1$, l=$I_q$] (0,5) -- (5,5) to[R=$R_3$, v=$U_3$] (7,5);
    \draw (2,0) to[short, *-] (2,1) to[R=$R_1$,v<=$U_1$] (2,4) to[short, -*] (2,5);
    \draw (0,0) -- (7,0);
    \draw (3.5,0) to[isource, i=$I_{q1}$, *-*] (3.5,5);
    \draw (5,0) to[short, *-] (5,1) to[R=$R_2$,v<=$U_2$] (5,4) to[short, -*] (5,5);
    \node[obj, label=above:A] (xa) at (7,5) {};
    \node[obj, label=below:B] (xa) at (7,0) {};
    \node[anchor=east] (Ua) at (7,2.5) {$U_{ab}$};
    \draw[-latex] (7, 3.5) -- (7, 1.5);
    \draw (3.5,0) to (3.5,-0.5) node[ground]{};
  \end{circuitikz}
\end{center}
Wir fassen nun die Stromquellen und die Widerstände zusammen und erhalten damit eine einzige Stromquelle mit einem einzigen Widerstand:
\begin{center}
  \begin{circuitikz}[european]
    \tikzstyle{obj}  = [circle, minimum width=5pt, fill, inner sep=0pt]
    \draw (0,0) to[isource, i=$I_1$, l=$I_q$] (0,5) -- (5,5) to[R=$R_3$, v=$U_3$] (7,5);
    \draw (2,0) to[short, *-] (2,1) to[R=$R_1$,v<=$U_1$] (2,4) to[short, -*] (2,5);
    \draw (0,0) -- (7,0);
    
    
    \node[obj, label=above:A] (xa) at (7,5) {};
    \node[obj, label=below:B] (xa) at (7,0) {};
    \node[anchor=east] (Ua) at (7,2.5) {$U_{ab}$};
    \draw[-latex] (7, 3.5) -- (7, 1.5);
    \draw (2,0) to (2,-0.5) node[ground]{};
  \end{circuitikz}
\end{center}
Da die Stromquellen unterschiedliche Richtungen hatten, wählen wir die Stromquelle $I_{q1}$ als positiven Strom und bekommen nun mit den Werten eine Stromquelle $I_q = 0,3A$ und $R_1 = \frac{80}{18} = 4,\bar{4}\Omega$.
Wir wandeln nun unsere Stromquelle zurück in eine Spannungsquelle und erhalten folgendes Bild:
\begin{center}
  \begin{circuitikz}[european]
    \tikzstyle{obj}  = [circle, minimum width=5pt, fill, inner sep=0pt]
    \draw (0,0) to[V, l=$U_q$] (0,2) to[R=$R_1$, v=$U_1$] (0,4) -- (0,5) -- (5,5) to[R=$R_3$, v=$U_3$] (7,5);
    \draw (0,0) -- (7,0);
    \node[obj, label=above:A] (xa) at (7,5) {};
    \node[obj, label=below:B] (xa) at (7,0) {};
    \node[anchor=east] (Ua) at (7,2.5) {$U_{ab}$};
    \draw[-latex] (7, 3.5) -- (7, 1.5);
    \draw (3.5,0) to[short, *-] (3.5,-0.5) node[ground]{};
  \end{circuitikz}
\end{center}
Dabei ist unser Wert für $U_q = 1,\bar{3}V$. Mit der folgenden Überlegung, bei der wir vereinfachend annehmen, ein unendlich großer Widerstand wäre zwischen Punkt A und B, ergibt sich ($R_L$ bezeichnet eben jenen Widerstand):
\begin{equation*}
  U_q = \lim_{R_L\to \infty} R_L\cdot I + R_1 \cdot I + R_3 \cdot I
\end{equation*}
Aufgrund des Ohm'schen Gesetzes, fließt bei einem unendlichen hohen Widerstand ein unendlich kleiner Strom, womit also gilt:
\begin{equation*}
  U_q = 1,\bar{3}V = \lim_{R_L\to \infty}\lim_{I\to 0} R_L\cdot I + R_1 \cdot I + R_3 \cdot I = \lim_{R_L\to \infty}\lim_{I\to 0} R_L\cdot I
\end{equation*}
Mit der obenstehenden Überlegung und der Potentialverlagerung in der Schaltung ist also zu erkennen, dass unsere Leerlaufspannung $U_{ab} = 1,\bar{3}V$ ist.

Da wir unsere Schaltung mithilfe von Äquivalenzumformung verändert haben, können wir den Innenwiderstand nun mithilfe der beiden verbleibenden Widerstände einfach berechnen über:
\begin{equation*}
  R_i = R_1 + R_3 = 5,\bar{4}\Omega 
\end{equation*}
Nun bleibt nur noch der Kurzschlussstrom übrig, den wir mit $I_k = \frac{U_q}{R_i} = \frac{1,\bar{3}V}{5,\bar{4}\Omega} = 0,24489A$ berechnen.
\subsection{Simulation}
\subsubsection{Durchführung}
Wie in den anderen Simulationen auch erstellen wir zunächst die Zeichnung der Aufgabe in LTSpice und geben zugleich die Werte für die Widerstände
und die Quellen ein. Danach definieren wir zur Übersicht noch die Punkte A und B.
Im Gegensatz zu den anderen Versuchen benutzen wir hier jedoch keine zeitabhängigen Variablen, um gewisse Verläufe zu simulieren, sondern benutzen lediglich die Operation Point Simulation Funktion von LTSpice,
sodass wir die Werte der Schaltung in einem stabilen Zustand der Schaltung erhalten.
\subsubsection{Beobachtung}
  \begin{center}
    \begin{tabular}{ c| c| c }
      & Wert & Einheit \\ 
     \hline
     V($U_{q0}$) & 23 & Volt\\
     V($U_{q1}$) & 20 & Volt \\
     V($U_{q2}$) & -16 & Volt \\
     V($\phi_0$) & 12,0459 & Volt \\
     V(a) & 14,456 & Volt \\  
     V(b) & 0 & Volt     \\
     I($R_1$) & 3,50574 & Ampere \\
     I($R_2$) & -1,09541 & Ampere \\
     I($R_3$) & 2,41 & Ampere \\
     I($R_4$) & -2,41 & Ampere
    \end{tabular}
    \end{center}
Wir erkennen anhand der Tabelle, dass unsere Berechnungen bezüglich dem Wert $U_{ab} = V(a) - V(b) = 14,456V$ mit unserem errechnetem Wert aus \textbf{Aufgabe 1.3.1}
übereinstimmt. Auch erkennen wir unsere Werte für die Ströme der Maschen, die wir im Zuge dessen berechnet hatten. 
Was jedoch auffällt, sind die Vorzeichenabweichungen bei den Stromwerten, die nicht in unserer Vorbereitung aufgetaucht sind. Zu erklären ist dies dadurch, dass wir bei unserer Berechnung vereinfachende Überlegungen aufstellen und anhand dieser rechnen,
wohingegen LTSpice die Schaltung so simuliert, wie sie in der realen Welt wäre, weshalb nicht alle Ströme in die gedachten Richtungen fließen. Da jedoch alle Beträge mit unseren Beträgen übereinstimmen, kommen sowohl die Simulation als unsere
Vorüberlegungen zu den gleichen Werten. 
\section{Auswertung}
Wie wir in den drei Versuchen gesehen haben, ist sowohl das Berechnen von Spannungs - und Stromquellen in Messbrückenschaltungen,
als auch das Simulieren und das Vorhersagen von Ergebnissen um einiges effizienter als bei einer beliebigen Schaltung. 
 
Mithilfe der Spannungs - bzw. Stromteilerregel lassen sich ohne das Aufstellen von komplexen
Gleichungssystemen oder Matrizen eine Berechnung der Werte mit hoher Genauigkeit und Schnelligkeit erreichen. Es ist daher von Vorteil, gerade bei Sensoren oder Schaltungen, die eine hohe Abhängigkeit von physikalischen Größen haben, eine solche Brückenschaltung zu verwenden.


Messbrückenschaltungen stellen die einfachste Möglichkeit dar, nicht-elektrische Größen wie Temperatur oder Masse in eine messbare elektrische Größe zu verwandeln.
Ein temperaturabhängiger Widerstand (z.B. PT100) verändert je nach der Umgebungstemperatur seinen Widerstand und lässt somit über die Verbindung beider Brückenleitungen eine entsprechende Spannung entstehen. 
Deshalb ist gerade das Messen von kleinen Widerstandsänderungen einfacher, wenn man sich in einer abgeglichenen Brückenschaltung befindet.

Es sei außerdem gesagt, dass eine Brückenschaltung zwar auch durch Geräte, wie z.B. einem Verstärker ersetzt werden könnte, diese jedoch wesentlich teurer sind, als eine leichte Umstrukurierung der Schaltung sowie ggfs. das Hinzufügen einer oder mehrer Widerstände.
Eine solche Schaltung ist wegen des Abgleichs beider Seiten nicht nur fehlerunanfälliger beim Messen, sondern auch das Berechnen solcher Schaltungen lässt sich wesentlich besser in Algorithmen zusammenfassen,
sodass sowohl Mensch als auch Maschine eine einfachere Umrechnungen der elektrischen Größen in physikalische Bezüge haben.
\newpage
\section{Anhang}
\subsection{Beweis zu (\ref{eq:pmax})}
\begin{lemma}
  Sei $\psi\text{ eine Abbildung}, \lambda, \mu \in \mathbb{R}$, wobei $\psi_n(\lambda, \mu) = \lambda \cdot \mu \land \lambda + \mu = x \in \mathbb{R} \text{ für beliebiges }n\in\mathbb{N}$ ist. Es folgt, dass der Hochpunkt der Funktion an der Stelle ist, an der $\lambda - \mu = 0$ gilt.
\end{lemma}
\begin{proof}
  Wir wollen nun zunächst den Hochpunkt von $\psi_n$ berechnen. Dafür stellen wir $\lambda + \mu = x$ nach $\lambda$ um und erhalten $\lambda = x - \mu$.
  Im nächsten Schritt setzen wir die umgestellte Gleichung in unsere Funktion ein und erhalten $\psi(\mu) = \mu \cdot (x - \mu) = x\mu - \mu^2$. Um den Extrempunkt der Funktion zu erhalten, leiten wir zunächst die Funktion einmal ab:
  \begin{equation*}
    \psi'(\mu) = x-2\mu
  \end{equation*}
  Setzen wir die Funktion nun gleich Null, ergibt sich:
  \begin{equation*}
    \begin{aligned}
    0 &= x - 2\mu \\
    2\mu &= x \\
    \mu &= \frac{x}{2}
    \end{aligned}
  \end{equation*}
  Leiten wir nun die Funktion ein weiteres Mal ab und setzen unser $\mu$ von oben ein, erhalten wir:
  \begin{equation*}
    \psi''\left(\frac{x}{2}\right) = -2 < 0 \implies \mu\text{ ist Hochpunkt der Funktion}
  \end{equation*}
  Dadurch, dass dieser Vorgang analog mit $\lambda$ funktioniert, und das gleiche Ergebnis $\frac{x}{2}$ entsteht, ist das Lemma gezeigt.
\end{proof}
\subsection{Zeichnung zu der Widerstandsberechnung von 1 und 2}
\begin{center}
  

\begin{circuitikz}[european]
  \draw (0, 1) to[short, -*] (1,1) -- (1,2) to[R=$R_1$] (4,2) to[R=$R_3$] (7,2) to[short, -*] (7,1) -- (8,1);
  \draw (1,1) -- (1,0) to[R=$R_2$] (4,0) to[R=$R_4$] (7,0) to[short] (7,1);
  \draw[color=red] (4,2) to[short, *-*] (4,0);
  \node[anchor=south] (xa) at (0,1) {a};
  \node[anchor=south] (xb) at (8,1) {b};
\end{circuitikz}
\end{center}
Anmerkung: Für Aufgabe 1 ist die rote Verbindungslinie vorhanden, während bei Aufgabe 2 die Linie nicht mehr da ist.
  
\end{document}